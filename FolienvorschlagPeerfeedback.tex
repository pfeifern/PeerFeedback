\documentclass{beamer}
\usepackage{hyperref}
 \usepackage[ngerman]{babel}
\setbeamertemplate{navigation symbols}{}
\usepackage{pstricks}
\makeatletter

\title{{\black Folienvorschlag zum}\\[1em]
  Peer-Feedback\\ zur Steigerung von Vortragskompetenzen}
\author[shortname]{Niki Pfeifer\inst{1} \and \& \and Thomas Neger\inst{2}}
\institute[shortinst]{\inst{1} Institut f\"ur  Philosophie\\Universit\"at
Regensburg\\\href{mailto:niki.pfeifer@ur.de}{\tt niki.pfeifer@ur.de}
\and \inst{2} Zentrum f\"ur Hochschul- und Wissenschaftsdidaktik  \\
Universit\"at Regensburg \\\href{mailto:thomas.neger@paedagogik.uni-regensburg.de}{\tt thomas.neger@paedagogik.uni-regensburg.de}}
\renewcommand{\blue}{\usebeamercolor[fg]{frametitle}} 
\begin{document}

\frame{\titlepage}

\frame{{Psychologische Wirkung bei Referaten}
{\blue Drei Ausdrucksebenen}
\begin{enumerate}
\item {\blue verbaler} Ausdruck \pause (Wort) 
\begin{itemize}
\item Wortwahl (Eloquenz)
\item Satzbau (einfach: z.B. Haupt- \& Nebensatz, Haupt- \& Hauptsatz)
\item Satzverbindungen
\item  Partikel (z.B. "`\"ahm"'), Floskeln (z.B. "`sozusagen"')\pause
\end{itemize}
\item {\blue paraverbaler} Ausdruck \pause (Stimme)
\begin{itemize}
\item Stimme  (freundlich bestimmt)
\item Betonung 
\item Tempo / Pausen (bis zu 7 Sekunden!)
\item Artikulation\pause
\end{itemize}
\item {\blue nonverbaler} Ausdruck \pause (K\"orper)
\begin{itemize}
\item Mimik
\item Blickkontakt
\item Gestik
\item K\"orperhaltung (sensomotorische R\"uckkoppelung)
\end{itemize}
\end{enumerate}
}

\frame{{Feedback}
\begin{description}
\item[\dots geben:] ~\\
\begin{itemize}
\item positive und negative Aspekte des Dargebotenen aufzeigen
\item konkret-beschreibend, nicht pauschal-interpretierend formulieren
\item m\"ogliche Konsequenzen des Verhaltens benennen
\item Kritik konstruktiv formulieren
\item pers\"onliche Stellungnahme statt "`man-Botschaften"'\pause
\end{itemize}
\item[\dots nehmen:] ~\\
\begin{itemize}
\item zuh\"oren ohne sich zu rechtfertigen
\item nachfragen, wenn etwas nicht verstanden wurde
\item die Bedeutung des Feedbacks f\"ur sich beschreiben
\end{itemize}
\end{description}
}
  
\frame{{Ablauf pro Referat} 
\begin{enumerate}
\item {\blue Referat} \pause
\item {\blue Verbalfeedback} (Notizen machen!) in folgender Reihenfolge:
\begin{enumerate}
\item Expertinnen- bzw. Expertenfeedback\label{gf}
\item ggf.\ Gruppenfeedback
\item ggf.\ Dozentenfeedback \pause
\end{enumerate}
\item {\blue Paraverbalfeedback} 
\item {\blue Nonverbalfeedback}
\item {\blue ggf. allgemeine Eindr\"ucke} (z.B. inhaltliche Aufbereitung, Gestaltung der
  Folien, usw.) \pause
\item {\blue Danksagung} durch Referierende und {\blue ggf.\ Stellungnahme} (z.B.:
  bewusst gesetzte Vortragsstrategien; wie haben Sie sich gef\"uhlt?), jedoch keine Rechtfertigung 
\end{enumerate}
}
\end{document}